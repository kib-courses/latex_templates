\documentclass{beamer}
\usetheme{metropolis} 
\usecolortheme{rose}

\hypersetup{unicode=true}

\usepackage{xcolor}
\usepackage[utf8]{inputenc}
\usepackage{hyphenat}
\usepackage[russian,english]{babel}          % Use metropolis theme
\usepackage{wrapfig}

\usepackage[normalem]{ulem}  % для зачекивания текста

\usepackage{caption}
\captionsetup[figure]{name=Рисунок }
\newcommand{\рис}[1]{рис.\ref{#1}}
\newcommand{\Рис}[1]{Рис.\ref{#1}}


\captionsetup[table]{name=Таблица~№}
\newcommand{\таблицa}[1]{таблица~№\ref{#1}} % именительный падеж
\newcommand{\таблицы}[1]{таблицы~№\ref{#1}} % родительный падеж
\newcommand{\таблице}[1]{таблице~№\ref{#1}} % дательный и предложный падеж
\newcommand{\таблицу}[1]{таблицу~№\ref{#1}} % винительный падеж
\newcommand{\таблицей}[1]{таблицей~№\ref{#1}} % творительный падеж 
\newcommand{\Таблицa}[1]{Таблица~№\ref{#1}} % именительный падеж
\newcommand{\Таблицы}[1]{Таблицы~№\ref{#1}} % родительный падеж
\newcommand{\Таблице}[1]{Таблице~№\ref{#1}} % дательный и предложный падеж
\newcommand{\Таблицу}[1]{Таблицу~№\ref{#1}} % винительный падеж
\newcommand{\Таблицей}[1]{Таблицей~№\ref{#1}} % творительный падеж 

\setbeamertemplate{footline}[frame number] % указывает на каждой странице общее количество страниц

% Указывайте все новые термины в \termdef команде. А уже известные ранее или из других курсов в \term
\newcommand{\termdef}[1]{\textbf{\textit{#1}}}
\newcommand{\term}{\textit}
% Диалог с аудиторией.
\newcommand{\auditorium}[1]{\color{red}{\textbf{#1}}}

% \setbeamercolor{auditorium}{fg=red}

% %%%%%%%%%%%%%%%%%%%%%%%%%%%%
% %%%%%%%%%%%%%%%%%%%%%%%%%%%%
% %%%%%%%%%%%%%%%%%%%%%%%%%%%%


% TODO ! напишите здесь название вашей лекции
\title{Лекция XX. Название вашей лекции.}
% TODO ! замените на дату проведения этой лекции. Например \date{14 апреля 2019}
\date{14 жерминаля пятого года Великой Революции}
\author{Максимилиан Максимилианович Робеспьер}
\institute{Москва, МГТУ им.Бауманка,\\ \href{https://t.me/iu8info}{\textbf{КИБ}}}
% \titlegraphic{\includegraphics[width=2cm]{logo_ur.jpg}}

% TODO ! замените https://github.com/kib-courses/latex_templates на ссылку ВАШЕГО спецкурса!
\titlegraphic{\small \href{https://github.com/kib-courses/latex_templates}{Название вашего спецкурса}}

\begin{document}
  \maketitle
    
  \begin{frame}{План лекции}
  	% TODO ! добавте в план все ваши секции, кроме "Вопросы для самопроверки", "Домашнее задание" и "Список материалов"
    \begin{enumerate}
	\item \nameref{section:main_latex_opportunity}
	\end{enumerate}
 \end{frame}
    
    
\section{Основные возможности \LaTeX}\label{section:main_latex_opportunity}

\begin{frame}{Формулы}\label{frame:example1}
   \LaTeX -- это удобно!
   Нужна
   пробельная
   строка,
   чтобы
   начать новый абзац!
   
   Можно делать формулы внутри текста: $(x+y)^2=x^2 + 2 \cdot x \cdot y + y^2$ и дальше писать
   текст.
   
   А можно снаружи:
   \begin{equation}\label{eq:example1}
   	f(x) = \frac{A_0}{2} + \sum \limits_{n=1}^{\infty} A_n \cos \left( \frac{2 n \pi x}{\nu} - \alpha_n \right) 
   \end{equation}
   
   И ссылаться на формулу: \eqref{eq:example1}, 
   которая находится на слайде №\ref{frame:example1}.  
\end{frame}

\begin{frame}{Редактирование текста}
	Текст может быть 
	\textbf{жирным}, 
	\textit{курсивным},
	\underline{подчёркнутым},
	\sout{зачёркнутым},
	\textcolor{red}{красного цвета}, \textcolor{green}{зелёного} 	
	и даже \textcolor[RGB]{18,10,143}{цвета ультрамарин}! 
	
	Текст может быть:
	\Huge очень огромным,
	\huge огромным,
	\Large очень большим,
	\large большим,
	\normalsize нормальным.
	
	Так же можно сделать шрифт
	\small маленьким,
	\footnotesize размером с ссылку,
	\tiny или крошечным.
	
	Можно так же \fbox{выделять текст рамкой}.
	\normalsize
	А этот текст нормальный, т.к. в исходниках выше этой строки есть \textit{\\normalsize} команда.
\end{frame}

\begin{frame}{Изображения}
	\begin{wrapfigure}{r}{5cm}
		\includegraphics[width=2cm]{../pic/kib_old_logo.png}
		\caption{Логотип в виде пеликана, кормящего грудью своих детей}
		\label{figure:pelican_1}
	\end{wrapfigure}
	Вы можете добавлять изображения.
	Текст при этом будет обрамляться.
	
	Ссылаться на изображения так же можно: \рис{figure:pelican_1}.
	
	Не забываем добавлять изображения в pic папку.
	Лучше использовать векторную графику. 
	Из растровой: png.
\end{frame}

\begin{frame}{Списки}
	Списки могут быть нумерованными. цифры(по умолчанию) или буквы. Например \textbf{[a]}:
	\begin{enumerate}[a]
		\item~ один
		\item~ два
		\item~ три
	\end{enumerate}
	и ненумерованными:
	\begin{itemize}
		\item~ один
		\item~ два
	\end{itemize}
	Их можно вкладывать друг в друга:
	\begin{enumerate}
		\item Первый пункт:
		\begin{enumerate}
			\item первый подпункт
		\end{enumerate}
		\item второй пункт.
	\end{enumerate}
\end{frame}

\begin{frame}
	Сам фрейм может не иметь заголовка. Как этот.
\end{frame}

\begin{frame}
	Можно делать таблицы:

	\begin{center}
	\small 
		\begin{table}
		\begin{tabular}{ l l }
			$0 \mapsto (3234, 25, 1, 1731) $ &  $0 \mapsto (2540, 55, 0, 1731)$ \\
			$1 \mapsto (18400, 45, 0, 3137)$ & $0 \mapsto (2540, 55, 0, 1731)$  \\
			$1 \mapsto (903, 19, 0, 4121)$  & $0 \mapsto (1875, 45, 0, 4121)$  \\
			$0 \mapsto (854, 21, 1, 4121)$  & $1 \mapsto (702, 21, 0, 4121)$  \\
			$1 \mapsto (903, 19, 0, 4121)$  & $0 \mapsto (1875, 45, 0, 4121)$  \\
			$0 \mapsto (28400, 41, 1, 3137)$ & $0 \mapsto (25040, 55, 0, 1731)$  \\
		\end{tabular}
		\caption{Это пример из второй лекции по DS спецкурсу}
		\label{table:example}
		\end{table}
	\end{center}
	И можно ссылаться на \таблицу{table:example}.
\end{frame}

 
\section{tikz. Основы}

\begin{frame}
Tikz -- это очень круто. Вот примеры: \url{http://www.texample.net/tikz/examples/all/}

Можно делать очень качественную векторную графику. 
А потом её вставлять в \LaTeX научные статьи.

Лучше их хранить в ОТДЕЛЬНОМ файле

\end{frame}


	


  
\end{document}